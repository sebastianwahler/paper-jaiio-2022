\todo{trabajos futuros - más descripción además de lo teórico}

Como continuación de este trabajo, existen diversas líneas de investigación que quedan abiertas y en las que es posible continuar trabajando. Durante el desarrollo de este trabajo han surgido algunas líneas futuras que se han dejado abiertas y que se esperan atacar en un futuro; algunas de ellas, están más directamente relacionadas con este trabajo y son el resultado de cuestiones que han ido surgiendo durante la realización del mismo.

A continuación se presentan algunas líneas de investigación futuras que pretendemos desarrollar como resultado de esta investigación o que también, pueden servir para ser retomadas posteriormente para otros investigadores. 

Entre los posibles trabajos futuros se destacan:

\subsubsection{Estudio de Centralidad de Intermediación (Betweenness centrality)}
Se pretende profundizar sobre el hecho de que 
Obtiene la medida en que un nodo en particular se encuentra entre otros nodos en una red. Estos elementos intermedios pueden ejercer control estratégico e influencia sobre muchos otros. El problema central de esta medida de centralidad es que un actor es central si se encuentra a lo largo de los caminos más cortos que conectan otros pares de nodos. Un individuo con una alta intermediación puede ser quien dirige la red. Un delincuente de este tipo es muy buscado, ya que su aprensión puede desestabilizar una red criminal o incluso hacer que se destruya~\cite{ref_article32}.

\subsubsection{Estudio de Centralidad de Cercanía (Closeness centrality)} Es la inversa de la suma de los caminos más cortos (geodésicas) que conectan un nodo particular con todos los demás nodos de una red. La idea es que un delincuente es central si puede interactuar rápidamente con todos los demás, no solo con sus primeros vecinos~\cite{ref_article33}. En el contexto de las bandas delictivas, esta medida destaca aquellos actores con la distancia mínima entre sí, lo que les permite comunicarse más rápidamente que cualquier otra persona en la organización. Por esta razón, la adopción de la centralidad de cercanía es crucial para poner en evidencia dentro de la red a aquellos individuos que están más "cerca" de otros (en términos de datos en común en los Casos en los que se encuentran involucrados). Además, los valores altos de centralidad de cercanía en este tipo de redes considerarse como un indicador de la "capacidad" del actor para difundir información rápidamente a todos los demás actores de la red.

\subsubsection{Estudio de Centralidad de Vector Propio o Autovector (Eigenvector centrality)} Es otra forma de asignar la centralidad a un actor de la red basada en la idea de que si un nodo tiene muchos vecinos centrales, también debería ser central. Esta medida establece que la importancia de un nodo está determinada por la importancia de sus vecinos. En el contexto de las redes de telecomunicaciones por ejemplo, la centralidad del vector propio generalmente se considera como la medida de influencia de un nodo dado. Altos valores de centralidad de vector propio son alcanzados por actores que están conectados con vecinos de alta puntuación, que a su vez, heredaron tal influencia de sus vecinos de alta puntuación y así sucesivamente. Esta medida refleja una característica importante en redes de comunicación, pero será interesante ver si se pueden extraer ciertas medidas dentro de nuestro campo de estudio.

\subsubsection{Estudio de herramienta de visualización "neovis.js"} Biblioteca de visualización de gráficos de JavaScript para crear visualizaciones que se pueden incrustar en una aplicación web, con la posibilidad de crear visualizaciones de datos gráficos que utilizan los resultados de algoritmos gráficos como Centralidad por PageRank y detección comunitaria. Dichas interacciones y configuraciones con los algoritmos, no son posibles de manipular con Vis.js. 