\todo{hablar de vis.js, quizás agregar algo de código}

Para llevar a cabo la visualización del conjunto de datos obtenidos del análisis inteligente anteriormente descripto, se utilizó Vis.js ~\cite{ref_visjs}, una biblioteca o librería de visualización dinámica basada en lenguaje Javascript. La misma está diseñada para que sea fácil de usar, para manejar grandes cantidades de datos dinámicos y para permitir la manipulación y la interacción con los datos. La biblioteca consta de los componentes DataSet, Timeline, Network, Graph2d y Graph3d.

En nuestro caso particular utilizamos el componente "Network", que permite mostrar redes en grafos. La visualización es fácil de usar y admite formas, estilos, colores, tamaños, imágenes, etc. Funciona sin problemas en cualquier navegador moderno para hasta unos pocos miles de nodos y bordes. Para manejar una mayor cantidad de nodos, Network tiene soporte de agrupamiento. La red utiliza canvas HTML para la renderización.