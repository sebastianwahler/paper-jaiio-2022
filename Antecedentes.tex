Las organizaciones criminales son grupos que operan fuera de la ley, realizando actividades ilegales en beneficio propio y en detrimento de otros individuos o grupos sociales~\cite{ref_article1}. Pueden ser de diverso tamaño y cubrir áreas geográficas variadas, en muchos casos en conflicto con otras organizaciones similares. Una de las características particulares de este tipo de organizaciones es que, al estar enfocadas en actividades ilegales perseguidas por los organismos de seguridad pública, el anonimato y/o la discreción de sus miembros es de vital importancia. Esto requiere estudios de la información existente con el fin de identificar los criminales y realizar acciones apropiadas para la prevención del delito.

Los miembros de las organizaciones criminales tienen a su vez diversos grados de compromiso con cada una de ellas. En muchos casos los hechos criminales que son evidentes en la sociedad ocurren por individuos de baja jerarquía y responsabilidad en el grupo, motivados por la recompensa inmediata, las aspiraciones de ascenso y la reputación en su propio círculo de contactos. Asimismo, existen otros individuos de mayor jerarquía y responsabilidad en la organización criminal, que ostentan cualidades de liderazgo, intereses a largo plazo, y la constante preocupación por la conservación del poder para el beneficio personal y de la organización. Con frecuencia, son los individuos del primer grupo los que cometen delitos percibidos y registrados por las fuerzas policiales, mientras que los miembros del segundo grupo se mantienen con mayor discreción. Adicionalmente, las estructuras jerárquicas, la forma de operar, y la cultura inherente de sus realidades socio-económicas  imponen códigos propios que hacen difícil la identificación de la organización delictiva como un todo, con sus miembros y actividades relacionadas. Es aquí donde nuestro trabajo puede aportar un rol significativo.