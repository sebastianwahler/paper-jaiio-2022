
En la actualidad las actividades criminales habituales en una ciudad o región van desde hurtos y robos de poca importancia, hasta otros de mayor gravedad como amenazas, cibercrimen, abusos sexuales y  homicidios. Todos ellos son registrados de diferentes formas por las fuerzas de la ley, con datos de variada precisión que incluyen usualmente la tipificación del delito, los datos en tiempo y espacio, y en muchas ocasiones los autores correspondientes.

Toda esta información respalda los procesos de investigación judicial de cada caso, pero con el transcurso del tiempo constituyen una extensa base de conocimiento sobre la cual es posible extraer valiosa información para la prevención del delito y la búsqueda de la justicia. Por ejemplo, es posible identificar relaciones entre personas de acuerdo a un análisis transitivo de eventos criminales en tiempo y espacio que sugieren la conformación de bandas delictivas. Las relaciones de amistad o conveniencia entre diversos autores de actividades criminales también puede inferirse de los registros delictivos y es de extrema relevancia para la prevención del delito y la resolución de casos inconclusos.

En este trabajo es de especial interés la aplicación de estas técnicas y tecnologías utilizando los registros de actividades criminales de la Provincia de Chubut a través de la colaboración del Ministerio Público Fiscal de la provincia (parte del Poder Judicial con autonomía funcional para la investigación y persecución de conductas delictivas) y las instituciones que lo asisten.